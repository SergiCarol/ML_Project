\documentclass[11pt, a4paper,titlepage]{article}
\usepackage[utf8]{inputenc}


%Margins
\usepackage[a4paper, total={6in, 9in}]{geometry}

%Català 
\usepackage[T1]{fontenc}
\usepackage[utf8]{inputenc}
%\usepackage[catalan]{babel}

%Imatges i peu de imatge
\usepackage{graphicx}
\usepackage{caption}
\usepackage{bmpsize}

\usepackage{amsmath}
\usepackage{ amssymb }
\usepackage{subcaption}
\usepackage{wrapfig}
\usepackage{adjustbox}

%Incloure codi
\usepackage{listings}

%Eliminar espai abans del primer item en itemize
\usepackage{enumitem}
\setlist{nolistsep}

%€
\usepackage[official]{eurosym}

%Links
\usepackage{hyperref}

%Colors
\usepackage{color}
\usepackage{enumitem}

\usepackage{multicol}

\edef\restoreparindent{\parindent=\the\parindent\relax}
\usepackage[parfill]{parskip}
\restoreparindent


\usepackage{blindtext}
\hypersetup{%
    pdfborder = {0 0 0}
}

%\usepackage{biblatex}
%\addbibresource{refs.bib}


\title{ \Large Multivariate Analysis \\~\\  \huge \textbf{Study on Beer recipes}}
\author{Carol, Sergi \\ Cebollero Ruiz, Laura \\ Reichl, Sofia}
\date{\today}

\begin{document}

\maketitle


\tableofcontents

\newpage
\section{Introduction}
Beer is one of the oldest, most popular and consumed alcoholic drink. It is the result of the fermentation of the sugar present in different cereal grains or, in some cases, even rice. There are many types of beer, such as Ale, Ipa or Imperial Stout, and in each of these types there are thousands of different brands and flavours.

Nowadays, lots of people are taking a more active role in the beer world and instead of just consuming it they are also keen on producing their own homebrewed beer. 

Because of this, many supporting communities that share techniques and recipes on how to produce beer have appeared, one of them being \textit{Brewer's Friend}. This website has compiled thousands of homebrewed beer recipes about their composition, acidity, style, bitterness and more.

The aim of this project is to use all the recipes provided in this website and study them, applying the multivariate techniques seen during this course to see how they relate to each other, and also see if they can be classified and, in case they are, that the corresponding classification corresponds to the existing styles.

\section{Preprocessing}
% todo Explain each variable in the data

\section{Protocol of validation}

\section{Visualization}

\section{Clustering}

\section{Differences between training and test samples}

\section{Prediction model}

\section{Conclusions}

\end{document}
